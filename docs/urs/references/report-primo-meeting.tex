\documentclass[11pt]{article}
\topmargin -.5in
\textheight 9in
\oddsidemargin -.25in
\evensidemargin -.25in
\textwidth 7in
\usepackage{xcolor}
\usepackage{hyperref}
\newcommand{\rosso}[1]{\color[rgb]{1,0,0}{#1}}
\newcommand{\blu}[1]{\color[rgb]{0,0,1}{#1}}
\newcommand{\bianco}[1]{\color[rgb]{1,1,1}{#1}}
\newcommand{\verde}[1]{\color[rgb]{0,0.65,0.31}{#1}}
\newcommand{\giallo}[1]{\color[rgb]{0.96,0.67,0}{#1}}
\newcommand{\nero}[1]{\color[rgb]{0,0,0}{#1}}

\newcommand{\dom}[1]{\item \color[rgb]{0,0,0}{#1}}
\newcommand{\seg}[1]{\item[] \color[rgb]{0,0,1}{#1}}
\newcommand{\art}[1]{\item[] \color[rgb]{0,0.65,0.31}{#1}}
\newcommand{\cli}[1]{\item[] \color[rgb]{1,0,0}{#1}}
\newcommand{\amm}[1]{\item[] \color[rgb]{0.96,0.67,0}{#1}}
\newcommand{\capo}[1]{\item[] \color[rgb]{0,1,0}{#1}}

\begin{document}

% ========== Edit your name here
\author{}
\title{Report di un primo briefing sul progetto Repair Speed-App}
\maketitle

\section{Contesto}
Una piccola ditta esegue interventi di riparazione per la casa:
piccoli interventi di manutenzione elettrica, piccoli interventi di
idraulica e carpenteria (riparazione di mobili, pareti in
legno/cartongesso). Ha diversi dipendenti: Una segretaria, un
amministrativo, circa 30 artigiani suddivisi in idraulici,
elettricisti e carpentieri.
%%
Il flusso di lavoro della ditta \`e il seguente: Un cliente chiama la
segretaria e spiega il problema. La segretaria innanzi tutto apre
una nuova richiesta di intervento (un foglio prestampato), registra
i dati del cliente, parlando con il cliente individua il problema, e
sceglie il tipo di artigiano che si pu\`o occupare del problema, e
infine individua delle possibili date per eseguire il lavoro.  Se il
cliente accetta, sceglie il giorno e l'ora e la segretaria finalizza
il foglio. Chiusa la telefonata registra l'intervento da effettuare.

%%
L'artigiano ogni mattina prende il foglio dove sono descritti gli
interventi che deve effettuare e si reca a domicilio per il risolvere
il problema.
%%
Al termine di ogni intervento l'artigiano deve compilare un rapperto
descrivendo l'intervento e aggiungendo le spese effettuate e il tempo
necessario  a risolvere il problema.
%%
Ottenuta la relazione attestante l'avvenuta riparazione, un
amministrativo compila la fattura con i dati del cliente e la
spedisce, monitorando il pagamento e laddove necessario sollecitando
il pagamento.
%%

 Il cliente, il proprietario della ditta, sostiene che la gestione
 cartacea \`e molto onerosa e macchinosa. Vorrebbe un sistema
 informatico che snellisse le procedure eliminando (quasi)
 completamente la carta.

\section{Concetti Generali}

\subsection{Stakeholders}:
In questa sezione vengono descritti gli stakeholders, chi sono e quali
sono gli obiettivi.

\begin{itemize}
\item Cliente (Padrone Azienda),
\item Segretaria,
\item Amministrativo,
\item Artigiano(Elettricista, Idraulico, Falegname),
\item Cliente.
\end{itemize}

\subsection{Campioni}
In questa sottosezione inseriamo i dati dei campioni da intervistare.
Il Proprietario, La Segretaria, L'Amministrativo,
 L'Idraulico, un cliente vecchio.....

 \subsection{Tecniche}
In questa sezione descriviamo le tecniche che si adotteranno. Le
tecniche verranno poi approfondite nelle prossime sezioni.
\begin{itemize}
\item Interviste mirate a campioni di ogni stakeholder.
\item Questionario Anonimo, a risposta multipla, con poche domande e
  al pi\`u 5 risposte.
\item Analisi di  prodotti concorrenti
\end{itemize}  
%% \subsection{Requisiti iniziali}
%% \begin{itemize}
%%   \item[] La segretaria deve Compilare un \textsl{Preventivo per una Richiesta di Intervento}
%%   \item[] La segretaria deve Compilare una \textsl{Richiesta di Intervento} 
%%   \item[] La segretaria deve Gestire le agende degli artigiani
%%   \item[] L'artigiano deve poter gestire la sua agenda
%%   \item[] L'artigiano deve poter chiudere una Richiesta di intervento
%%   \item[] L'artigiano deve poter compilare  un rapporto per ogni intervento effettuato
%%   \item[] .....
%%   \item[]   
%%   \end{itemize}

\section{Tecniche di Elicitation: Interviste}
In questa sezione elenchiamo alcune domande che vorremmo sottoporre ai
vari stakeholder. Sono divise in sezioni, ed ogni sezione racchiude le
domande per un gruppo di persone (stakeholder).

\subsection{Domande per \blu{Segretaria}}
\begin{enumerate}
\dom{Quali sono i tuoi compiti?}
\dom{Che cos'è una richiesta d'intervento?}
\dom{Quali sono i dati del cliente? I dati della richiesta?}
\dom{Come fai a capire il tipo di intervento?}
\dom{Come si conclude la richiesta di intervento?}
\dom{Il cliente deve rispondere subito?}
\dom{Se il cliente accetta cosa succede?}
\dom{Entro quanto tempo deve rispondere?}
\dom{Se il cliente non accetta il preventivo cosa succede?}
\dom{Se il cliente concorda una data cosa succede?}
\dom{Quali sono i tuoi problemi in queste procedure?}
\dom{Perch\`e il capire il tipo di intervento \`e un problema?}
\dom{E come fai a capire l'intervento?}
\dom{E se nemmeno cos\`i si viene a capo?}
\dom{Parlami della gestione dell'agenda, come viene fatta?}
\dom{Ma per chi sono i fogli?}
\dom{Come vengono allocati gli interventi? Come si fanno a evitare sovrapposizioni? Quanto pu\`o durare un'intervento?}

\dom{Che differenza c'\`e tra un intervento normale e uno d`urgenza?.}
\dom{Cosa succede se l'intervento non andrà a buon fine}
\dom{Cosa succede se non ci sono artigiani disponibili?}
\dom{Quali sono gli interventi più generali?}
\dom{Quanto spesso non capisci il tipo di intervento?}
\dom{Cosa succede se non si fa la detection dell'intervento?}
\dom{Intendevo: se non riesce ad individuare il tipo di intervento da effettuare, che cosa succede?}
\dom{E se anche cos\`i non si riuscisse a individuare il tipo di intervento?}
\dom{Cosa succede se non ci sono artigiani da assegnare alla richiesta dell'intervento?}
\dom{Se è in corso un intervento normale, come viene gestito uno d'urgenza che viene richiesto nel frattempo e nessuno specializzato è libero?}
\dom{E se arriva una richiesta d'urgenza quando l'artigiano reperibile \`e impegnato?}
\dom{E se la coda diventa troppo lunga?}
\dom{E se non ne trova?}
\dom{Cosa accade se è richiesto un intervento di urgenza, ma è in corso lo stato di allerta meteo?}
\dom{Usi programmi come office?}
\dom{Qual'è il tuo livello di utilizzo?}
\dom{In base a cosa scegli l'artigiano?}
\dom{Come decidi la priorità per ogni riparazione?}
\dom{Se un artigiano fa male un intervento, e il cliente si lamenta, come gestisci la situazione?}
\dom{Se un artigiano crea un danno al cliente, come viene gestista la situazione?}
\dom{Si può modificare data, artigiano o altri campi dell'intervento? Esiste già nel caso una procedura a riguardo?}
\dom{Come viene comunicato un intervento urgente?}
\dom{Ci sono già domande precompilate per l'individuazione del tipo di intervento?}
\dom{Come si riallocano gli interventi in caso di problemi?}
\dom{E se un artigiano non pu\`o eseguire l'intervento? Ad esempio perch\`e un altro intervento ha richiesto pi\`u tempo, oppure malato, oppure incidente, oppure traffico (hanno chiuso le strade per arricarci)?}
\dom{Qual'è la procedura da seguire se il cliente richiama per modificare la data dell'intervento?}
\dom{Ma questo non influenza anche le agende degli altri artigiani dello stesso tipo?}
\dom{In una richiesta d'intervento pu\`o essere cambiato anche il tipo d'intervento? Se il cliente si accorgesse di aver sbagliato a descrivere il problema?}
\dom{Cosa succede se l'intervento richiede più di una visita?}
\dom{Cosa succede se il cliente richiama dopo alcuni giorni affermando che il lavoro non ha risolto definitivamente il problema?}
\dom{Ma il primo intervento viene fatturato?}
\dom{Cosa succede se il cliente non è in casa il giorno dell'appuntamento?}
\dom{Come viene gestita l'assegnazione di mezzi e strumenti aziendali ai vari artigiani?}
\dom{Come viene gestito il pagamento da parte dei clienti?}
\dom{Che durata e tipologia ha la garanzia dei vari interventi?}
\dom{Come sceglie l'operaio che deve intervenire?}
\dom{Come riceve la conferma dagli artigiani per l'intervento?}
\dom{Come tiene traccia degli appuntamenti?}
\dom{Come prevede la durata di un intervento?}
\dom{Se il cliente non è soddisfatto cosa succede?}
\dom{E' possibile rateizzare un pagamento?}
\dom{Per quanto tempo tenete traccia dei dati dei clienti?}
\dom{Fornite un'assistenza post-intervento?}
\dom{Su quali parametri viene calcolato il preventivo?}
\dom{Quanto tempo impiega a completare la richiesta di intervento?}
\dom{Viene mantenuto uno storico degli interventi effettuati ? Dopo quanto tempo vengono cancellati ?}
\end{enumerate}

\subsection{Domande per \verde{Artigiano}}
\begin{enumerate}
\dom{Quali sono i tuoi compiti?}
  \dom{In cosa consiste un intervento?}
  \dom{Cosa succede se l'intervento ha bisogno di pezzi aggiuntivi?}
  \dom{Cosa succede se la lista preventiva dei pezzi non \`e esaustiva?}
  \dom{A quel punto se il pezzo \`e indispensabile?}
  \dom{E se il pezzo non fosse disponibile a magazzino e ci volesse pi\`u tempo per arrivare?}
  \dom{Se l'intervento va a buon fine, cosa succede?}
  \dom{E che dati ci sono nella relazione?}
  \dom{Quanto dura un intervento in media?}
  \dom{Cosa succede se non finisci in tempo?}
  \dom{Cosa succede se un cliente ti ruba gli attrezzi? Devi avviare una controversia?}
  \dom{Cosa succede se perdi degli attrezzi?}
  \dom{Cosa succede se mentre lavori il cliente ti dice che devi cambiare qualcosa perchè stai sbagliando?}
  \dom{Come vieni contattato dalla segretaria?}
\dom{se non hai un pezzo, e la riparazione è urgente, puoi acquistarlo da solo?}
\dom{cosa succede se non trovi il cliente?}
\dom{Cosa succede se arrivato dal cliente ti accorgi che il problema non lo puoi risolvere ma ci vuole un altri tipo di artigiano?}
\dom{Si adebbita comunque la visione del problema? Come si rischedula?}
\dom{Se fosse richiesto un artigiano aggiuntivo (altro campo di competenze)?}
\dom{Se l'intervento non va a buon fine cosa succede?}
\dom{Quali possono essere i motivi per cui un intervento non va a buon fine?}
\dom{Se la segretaria sbaglia il tempo necessario per eseguire il lavoro, come gestisce una eventuale sovrapposizione?}
\dom{Cosa deve fare se non riesce a rispettare gli interventi della giornata?}
\dom{Cosa succede se arriva da un cliente e non deve intervenire?}
\dom{Come gestisce una controversia con il cliente?}
\dom{Come si comporta se al momento dell'intervento manca il materiale/componente corretto?}
\dom{Qual'è la procedura che si esegue nel caso di un'emergenza?}
\dom{Nel caso in cui un'urgenza debba sovrapporsi ad un intervento ordinario, come viene rischedulato?}
\dom{Puoi richiedere l'aiuto di un collega? Se si come si registra la sua collaborazione?}
\dom{Hai avuto difficoltà nel compilare il rapporto ? }
\dom{Cosa fai quando c'\`e un intervento?}
\dom{Quali sono i dati pi\`u importanti nella richesta?}
\end{enumerate}


\subsection{Domande per \rosso{Cliente} \nero{che apre una richiesta}}
\begin{enumerate}
\dom{\'E soddisfatto del servizio?}
\dom{Quali sono i problemi che ha riscontrato?}
\dom{\'E soddisfatto dei tempi del servizio?}
\dom{Vli piacerebbe prenotare un intervento tramite un applicazione?}
\dom{Ci sceglierebbe di nuovo per lavori futuri?}
\dom{Auale è il metodo di pagamento utilizzabile preferito?}
\dom{E' soddisfatto dalla tipologia di pagamento?}
\dom{Come valuti il rapporto qualità-prezzo?}
\dom{Cosa miglioreresti del servizio?}
\dom{Sei soddisfatto del tempo impiegato dall'artigiano per arriavare (tempo per arrivare a casa tua) a risolvere il problema?}
\dom{Sei soddisfatto del tempo impiegato dall'artigiano per risolvere il problema?}
\dom{Sei soddisfatto del tempo che la segretaria ha impiegato a capire quale fosse il problema?}
\dom{L'amministrativo, ha impiegato poco tempo per inviare la fattura?}
\dom{Sopo mesi dall'intervento hai riscontrato ancora lo stesso problema che avevi prima dell'intervento?}
\dom{La qualità dei pezzi è buona (non si sono rovinati dopo mesi dall'intervento)?}
\dom{Cosa cambierebbe nel servizio, ai fini di migliorarlo?}
\dom{Ha mai riscontrato errori nell'assegnazione degli artigiani in relazione al problema da lei esposto (idraulico, elettricista...) ?}
\dom{Preferirebbe vedere prima un listino prezzi?}
\dom{Vorrebbe poter scegliere l'artigiano?}
\dom{Vorrebbe poter commentare e leggere recensioni sugli artigiani prima di contattarli?}
\dom{In quale modo preferirebbe avere contatti con la ditta?(telefono, mail,piattaforma web)}
\dom{Che difficoltà ha avuto nello spiegare il problema?}
\dom{Ha avuto problemi con il pagamento?}
\dom{La fattura è arrivata nei giusti tempi?}
\dom{Richiameresti quest'azienda in futuro? perchè?}
\dom{Se non è soddisfatto del servizio cosa succede?}
\dom{Se il prezzo è alto cosa fa?}
\end{enumerate}

\section{Domande per Amministrativo}
\begin{enumerate}
  \dom{Quali sono i dati utilizzati per la fattura?}
  \dom{A quando risalgono le ultime fatture conservate?}
  \dom{Quali sono i metodi di pagamento disponibili al cliente?}
  \dom{Come viene monitorato il pagamento del cliente? dopo quanto vi è il sollecito?}
  \dom{Tutte le fatture da pagare sono tenute in un     raccoglitore. Periodicamente controllo sul conto dell'azienda se
    sono state pagate. Se sono state pagate le archivio, altrimenti faccio un sollecito.}
  \dom{Dopo quanti giorni di mancato pagamento viene inviata la prima sollecitazione?}
  \dom{Ci sono differenze di fatturazione fra interventi normali e interventi d'urgenza? }
  \dom{Se il cliente rifiuta il pagamento perché è tropo caro cosa succede?}
  \dom{Come prepara la fattura?}
  \dom{Come ci si comporta se un cliente non paga?}
  \dom{Come sono gestiti i pagamenti dei dipendenti?}
  \dom{Quali sono le problematiche che riscontri maggiormente?}
  \dom{Quali sono i tuoi compiti?}
  \dom{Che difficoltà hai trovato nella compilazione della fattura?}
  \dom{Ci sono problemi di passaggio della relazione di avvenuta riparazione tra la segreteria e il suo ufficio?}
  \dom{Come rilasciate le fatture?}
  \dom{Gli artigiani compilano sempre in modo adeguato i rapporti?}
  \dom{Nel caso di mancate informazioni come vi comportate?}
  \dom{Come gestite l'acquisto di nuovi pezzi, strumenti ecc...?}
  \dom{In caso un pezzo non fosse disponibile presso i soliti     fornitori o fosse disponibile da altri negozi in tempi/prezzi
    minori cosa di fa?}
  \dom{Ogni quanto vengono consegnate le relazioni riguardanti gli interventi?}
  \dom{Che tipologia di tecnologie informative sta attualmente utilizzando nel suo lavoro?}
\end{enumerate}

\section{Domande per Cliente}
\begin{enumerate}
\dom{\'E soddisfatto del servizio?}
\dom{Quali sono i problemi che ha riscontrato?}
\dom{\'E soddisfatto dei tempi del servizio?}
\dom{Vli piacerebbe prenotare un intervento tramite un applicazione?}
\dom{Ci sceglierebbe di nuovo per lavori futuri?}
\dom{Auale è il metodo di pagamento utilizzabile preferito?}
\dom{E' soddisfatto dalla tipologia di pagamento?}
\dom{Come valuti il rapporto qualità-prezzo?}
\dom{Cosa miglioreresti del servizio?}
\dom{Sei soddisfatto del tempo impiegato dall'artigiano per arriavare (tempo per arrivare a casa tua) a risolvere il problema?}
\dom{Sei soddisfatto del tempo impiegato dall'artigiano per risolvere il problema?}
\dom{Sei soddisfatto del tempo che la segretaria ha impiegato a capire quale fosse il problema?}
\dom{L'amministrativo, ha impiegato poco tempo per inviare la fattura?}
\dom{Sopo mesi dall'intervento hai riscontrato ancora lo stesso problema che avevi prima dell'intervento?}
\dom{La qualità dei pezzi è buona (non si sono rovinati dopo mesi dall'intervento)?}
\dom{Cosa cambierebbe nel servizio, ai fini di migliorarlo?}
\dom{Ha mai riscontrato errori nell'assegnazione degli artigiani in relazione al problema da lei esposto (idraulico, elettricista...) ?}
\dom{Preferirebbe vedere prima un listino prezzi?}
\dom{Vorrebbe poter scegliere l'artigiano?}
\dom{Vorrebbe poter commentare e leggere recensioni sugli artigiani prima di contattarli?}
\dom{In quale modo preferirebbe avere contatti con la ditta?(telefono, mail,piattaforma web)}
\dom{Che difficoltà ha avuto nello spiegare il problema?}
\dom{Ha avuto problemi con il pagamento?}
\dom{La fattura è arrivata nei giusti tempi?}
\dom{Richiameresti quest'azienda in futuro? perchè?}
\dom{Se non è soddisfatto del servizio cosa succede?}
\dom{Se il prezzo è alto cosa fa?}
\end{enumerate}


\section{Tecniche di Elicitation: Questionario}
In questa sezione dobbiamo elencare i tipi di questionario che si
vogliono somministrare, indicando principalmente l'obiettivo e le
domande/risposte. In particolare vogliamo perseguire due obiettivi:
(i) capire il grado di alfabetizzazione informatica di tutti gli
stakeholder e (ii) capire il grado di conoscenza dei problemi dei
clienti.  Quindi prepariamo due questionari, ogni questionario deve
avere tra le 5 e le 10 domande, e ogni domanda deve avere esattamente
5 risposte, tra cui non lo so, e altro.

\subsection{Questionario alfabetizzazione informatica}
Il questionario ha come obiettivo quello di capire come i dipendenti
dell'azienda si rapportano con il computer.  \`E diviso in due parti,
nella prima ci sono domande di anagrafica (et\`a, Famiglia, possiedi
pc), e una seconda parte domande a carattere generale sul PC (Qual'\`e
il tuo grado di conoscenza di Office?)
\href{http://www00.unibg.it/dati/bacheca/313/9143.pdf}{Esempio di Questionario}
%% \begin{enumerate}
%% \item
%%   \begin{enumerate}
%%   \item 
%%   \end{enumerate}
%% \end{enumerate}  


\section{Valutazione di applicazioni equivalenti}
In questa sezione descriviamo uno o pi\`u applicazioni che possono
essere d'interesse e avere funzionalit\`a in comune con l'applicazione
che si vuole sviluppare. In ogni sottosezione indicare il link e le
funzionalit\`a spiegandone le caratteristiche.




\end{document}
