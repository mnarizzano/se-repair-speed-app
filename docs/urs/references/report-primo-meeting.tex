\documentclass[11pt]{article}
\topmargin -.5in
\textheight 9in
\oddsidemargin -.25in
\evensidemargin -.25in
\textwidth 7in
\usepackage{xcolor}
\usepackage{hyperref}
\newcommand{\rosso}[1]{\color[rgb]{1,0,0}{#1}}
\newcommand{\blu}[1]{\color[rgb]{0,0,1}{#1}}
\newcommand{\bianco}[1]{\color[rgb]{1,1,1}{#1}}
\newcommand{\verde}[1]{\color[rgb]{0,0.65,0.31}{#1}}
\newcommand{\giallo}[1]{\color[rgb]{0.96,0.67,0}{#1}}
\newcommand{\nero}[1]{\color[rgb]{0,0,0}{#1}}

\newcommand{\dom}[1]{\item \color[rgb]{0,0,0}{#1}}
\newcommand{\seg}[1]{\item[] \color[rgb]{0,0,1}{#1}}
\newcommand{\art}[1]{\item[] \color[rgb]{0,0.65,0.31}{#1}}
\newcommand{\cli}[1]{\item[] \color[rgb]{1,0,0}{#1}}
\newcommand{\amm}[1]{\item[] \color[rgb]{0.96,0.67,0}{#1}}
\newcommand{\capo}[1]{\item[] \color[rgb]{0,1,0}{#1}}

\begin{document}

% ========== Edit your name here
\author{}
\title{Report di un primo briefing sul progetto Repair Speed-App}
\maketitle

\section{Contesto}
Una piccola ditta esegue interventi di riparazione per la casa:
piccoli interventi di manutenzione elettrica, piccoli interventi di
idraulica e carpenteria (riparazione di mobili, pareti in
legno/cartongesso). Ha diversi dipendenti: Una segretaria, un
amministrativo, circa 30 artigiani suddivisi in idraulici,
elettricisti e carpentieri.
%%
Il flusso di lavoro della ditta \`e il seguente: Un cliente chiama la
segretaria e spiega il problema. La segretaria innanzi tutto apre
una nuova richiesta di intervento (un foglio prestampato), registra
i dati del cliente, parlando con il cliente individua il problema, e
sceglie il tipo di artigiano che si pu\`o occupare del problema, e
infine individua delle possibili date per eseguire il lavoro.  Se il
cliente accetta, sceglie il giorno e l'ora e la segretaria finalizza
il foglio. Chiusa la telefonata registra l'intervento da effettuare.

%%
L'artigiano ogni mattina prende il foglio dove sono descritti gli
interventi che deve effettuare e si reca a domicilio per il risolvere
il problema.
%%
Al termine di ogni intervento l'artigiano deve compilare un rapperto
descrivendo l'intervento e aggiungendo le spese effettuate e il tempo
necessario  a risolvere il problema.
%%
Ottenuta la relazione attestante l'avvenuta riparazione, un
amministrativo compila la fattura con i dati del cliente e la
spedisce, monitorando il pagamento e laddove necessario sollecitando
il pagamento.
%%

 Il cliente, il proprietario della ditta, sostiene che la gestione
 cartacea \`e molto onerosa e macchinosa. Vorrebbe un sistema
 informatico che snellisse le procedure eliminando (quasi)
 completamente la carta.

\section{Concetti Generali}

\subsection{Stakeholders}:
In questa sezione vengono descritti gli stakeholders, chi sono e quali
sono gli obiettivi.

\begin{itemize}
\item Cliente (Padrone Azienda),
\item Segretaria,
\item Amministrativo,
\item Artigiano(Elettricista, Idraulico, Falegname),
\item Cliente.
\end{itemize}

\subsection{Campioni}
In questa sottosezione inseriamo i dati dei campioni da intervistare.
Il Proprietario, La Segretaria, L'Amministrativo,
 L'Idraulico, un cliente vecchio.....

 \subsection{Tecniche}
In questa sezione descriviamo le tecniche che si adotteranno. Le
tecniche verranno poi approfondite nelle prossime sezioni.
\begin{itemize}
\item Interviste mirate a campioni di ogni stakeholder.
\item Questionario Anonimo, a risposta multipla, con poche domande e
  al pi\`u 5 risposte.
\item Analisi di  prodotti concorrenti
\end{itemize}  
%% \subsection{Requisiti iniziali}
%% \begin{itemize}
%%   \item[] La segretaria deve Compilare un \textsl{Preventivo per una Richiesta di Intervento}
%%   \item[] La segretaria deve Compilare una \textsl{Richiesta di Intervento} 
%%   \item[] La segretaria deve Gestire le agende degli artigiani
%%   \item[] L'artigiano deve poter gestire la sua agenda
%%   \item[] L'artigiano deve poter chiudere una Richiesta di intervento
%%   \item[] L'artigiano deve poter compilare  un rapporto per ogni intervento effettuato
%%   \item[] .....
%%   \item[]   
%%   \end{itemize}

\section{Tecniche di Elicitation: Interviste}
In questa sezione elenchiamo alcune domande che vorremmo sottoporre ai
vari stakeholder. Sono divise in sezioni, ed ogni sezione racchiude le
domande per un gruppo di persone (stakeholder).

\subsection{Domande per \blu{Segretaria}}

\begin{enumerate}
\dom{Quali sono i tuoi compiti?}
\seg{Rispondere al telefono e raccogliere le richieste di intervento.}
\dom{Che cos'è una richiesta d'intervento?}
\seg{\`E un foglio dove sono registrati i dati del cliente e i dati della richiesta.}
\dom{Quali sono i dati del cliente? I dati della richiesta?}
\seg{... e dove \`e descritto il tipo di intervento. ....}
\dom{Come fai a capire il tipo di intervento?}
\seg{Sulla base della mia esperienza e tramite descrizione
  con il cliente, alla fine il tipo di intervento viene registrato
  sulla richiesta e concordato con il cliente.}
\dom{Come si conclude la richiesta di intervento?}
\seg{Devo fare un preventivo e chiedere al cliente se per lui va bene.}
\dom{Il cliente deve rispondere subito?}
\seg{No, dopo una settimana ogni richiesta non accettata scade.}
\dom{Se il cliente accetta cosa succede?}
\seg{Guardo le agende degli artigiani e propongo alcune date al cliente.}
\dom{Entro quanto tempo deve rispondere?}
\seg{Dopo una settimana il preventivo non \`e pi\`u valido, salvo accordi con il cliente.}
\dom{Se il cliente non accetta il preventivo cosa succede?}
\seg{Il preventivo viene cestinato.}
\dom{Se il cliente concorda una data cosa succede?}
\seg{Registro la data sulla richiesta d'intervento, aggiorno le agende degli artigiani.}
\dom{Quali sono i tuoi problemi in queste procedure?}
\seg{Capire il tipo di intervento, fare un preventivo sensato, recuperare i dati dei clienti, la gestione dell'agenda.....}
\dom{Perch\`e il capire il tipo di intervento \`e un problema?}
\seg{Perc\`e non sono ne un idraulico, ne un elettricista e nemmeno un carpentiere. Inoltre i problemi sono molto diversi.}
\dom{E come fai a capire l'intervento?}
\seg{Principalmente tramite l'esperienza. Se non ricordo a
    memoria, vado a scartabellare tra gli interventi passati che sono
    catalogati per tipologia e poi discuto con il cliente.}
\dom{E se nemmeno cos\`i si viene a capo?}
\seg{Attacco, chiamo un esperto e cerco di farmi un idea e poi richiamo il cliente.}
\dom{Parlami della gestione dell'agenda, come viene fatta?}
\seg{O mamma, un vero incubo. Per ogni mese ho un foglio per
  ciascun artigiano dove sono segnati i giorni e nei giorni i suoi
    impegni. Inoltre Per ogni categoria ho un foglio che rappresenta
    il riassunto del mese. Quando devo aggiungere un nuovo intervento
    prima guardo il riassunto e poi i singoli.}
\dom{Ma per chi sono i fogli?}
\seg{I fogli nominativi sono per gli artigiani, i riassunti
    per me per evitare di perdermi. Recentemente ho cominciato a
    registare i fogli di riassunto su un file excel.}
\dom{Come vengono allocati gli interventi? Come si fanno a
    evitare sovrapposizioni? Quanto pu\`o durare un'intervento?}

\seg{Noi assumiamo che per eseguire un'intervento ci
    vogliano circa 2 ore, quindi ne scheduliamo 2 al mattino e due al
    pomeriggio. Come norma per\`o ne scheduliamo tre in modo da
    lasciare spazio per le urgenze.}
\dom{Che differenza c'\`e tra un intervento normale e uno d`urgenza?.}
\seg{Costi e tempi di risposta diversi.}

\dom{Cosa succede se l'intervento non andrà a buon fine}

\dom{Cosa succede se non ci sono artigiani disponibili?}
\seg{Si cambia giorno fino a quando non si trova un artigiano disponibile.}
\dom{Quali sono gli interventi più generali?}
\dom{Quanto spesso non capisci il tipo di intervento?}
\seg{All'inizio era molto frequente, ma con il passare del tempo sono diventata pi\`u esperta. Ci sono giorni che non capita mai, altri che capita una volta. Nell'ultimo anno non ricordo che sia mai capitato pi\`u di una volta al giorno.}
\dom{Cosa succede se non si fa la detection dell'intervento?}
\seg{Scusi pu\`o ripetere?}
\dom{Intendevo: se non riesce ad individuare il tipo di intervento da effettuare, che cosa succede?}
\seg{Ah beh, metto in attesa il cliente, e chiamo l'artigiano che mi sembra possa essere assegnato all'intervento e con lui discuto. Continuo a chiamare fino a che non mi risponde qualcuno che mi cataloga l'intervento come uno degli interventi che pu\`o effettuare la sua categoria.}
\dom{E se anche cos\`i non si riuscisse a individuare il tipo di intervento?}
\seg{Non \`e mai capitato. Immagino per\`o che ci sia bisogno di un sopralluogo.}
\dom{Cosa succede se non ci sono artigiani da assegnare alla richiesta dell'intervento?}
\seg{Si cambia giorno.}
\dom{Se è in corso un intervento normale, come viene gestito uno d'urgenza che viene richiesto nel frattempo e nessuno specializzato è libero?}
\seg{Sono da considerarsi d'urgenza gli interventi effettuati entro le 12 ore dalla chiamata. Tipicamente un artigiano per tipo \`e reperibile ogni giorno per gli interventi d'urgenza.}
\dom{E se arriva una richiesta d'urgenza quando l'artigiano reperibile \`e impegnato?}
\seg{Si mette in coda la richiesta.}
\dom{E se la coda diventa troppo lunga?}
\seg{A questo punto chiamo gli altri artigiani di quel tipo che non sono impegnati e ne cerco uno da aggiungere a quelli reperibili.}
\dom{E se non ne trova?}
\seg{Metto le richieste in coda.}
\dom{Cosa accade se è richiesto un intervento di urgenza, ma è in corso lo stato di allerta meteo?}
\seg{Solo in caso di allerta rossa rischeduliamo tutti gli appuntamenti non urgenti chiamando il cliente.}
\dom{Usi programmi come office?}
\seg{Si}
\dom{Qual'è il tuo livello di utilizzo?}
\seg{Riesco a utilizzare bene word ed excel con multitabelle. Riesco anche a inserire formule nelle celle per eseguire dei calcoli.}
\dom{In base a cosa scegli l'artigiano?}
\seg{In base al tipo d'intervento, e in base ai giorni richiesti dal cliente.}
\dom{Come decidi la priorità per ogni riparazione?}
\seg{Il primo che arriva sceglie.}
\dom{Se un artigiano fa male un intervento, e il cliente si lamenta, come gestisci la situazione?}
\seg{Avviso il padrone che si fa carico di controllare il lavoro.}
\dom{Se un artigiano crea un danno al cliente, come viene gestista la situazione?}
\seg{Se il danno viene riconosciuto dall'artigiano allora ci occupiamo di risarcire il cliente tramite un'assicurazione che abiamo stipulato.}
\dom{Si può modificare data, artigiano o altri campi dell'intervento? Esiste già nel caso una procedura a riguardo?}
\seg{Si pu\`o modificare la data dell'intervento fino a 12 ore prima dell'intervento stesso. Il cliente deve chiamare in segreteria.}
\dom{Come viene comunicato un intervento urgente?}
\seg{L'urgenza la stabilisce il cliente, se vuole che l'intervento venga eseguito entro la giornata (entro 6/8 ore) allora deve comunicarlo a me mentre siamo al telefono.}
\dom{Ci sono già domande precompilate per l'individuazione del tipo di intervento?}
\seg{Si}
\dom{Come si riallocano gli interventi in caso di problemi?}
\seg{Se il problema avviene a seguito del primo intervento, con il cliente si concorda un'altra data in cui finire l'intervento. Per la riallocazione bisogna sentire le parti interessate (cliente, magazzino per eventuali pezzi e artigiano).}
\dom{E se un artigiano non pu\`o eseguire l'intervento? Ad esempio perch\`e un altro intervento ha richiesto pi\`u tempo, oppure malato, oppure incidente, oppure traffico (hanno chiuso le strade per arricarci)?}
\seg{Se il cliente \`e raggiungibile si cerca di allocare l'intervento gli altri artigiani di quel tipo che lavorano in quella giornata e siano parzialmente scarichi. Se tutti gli artigiani sono a tappo e se non ci sono troppe urgenze si utilizzano gli artigiani reperibili, altrimenti si chiama il cliente e si concorda un altro giorno.}
\dom{Qual'è la procedura da seguire se il cliente richiama per modificare la data dell'intervento?}
\seg{Se fatta entro le 12 ore precedenti l'intervento semplicemente si cancella l'appuntamento e si procede a concordare con il cliente una nuova data.}
\dom{Ma questo non influenza anche le agende degli altri artigiani dello stesso tipo?}
\seg{Non necessariamente. L'agenda viene creata la sera precedente in base agli interventi e le aree geografiche. Per cui se una persona chiama il giorno prima le agende non sono ancora pronte.}
\dom{In una richiesta d'intervento pu\`o essere cambiato anche il tipo d'intervento? Se il cliente si accorgesse di aver sbagliato a descrivere il problema?}
\seg{Se fatta entro le 12 ore precedenti l'intervento semplicemente si cancella l'appuntamento e si procede a concordare con il cliente una nuova data.}
\dom{Cosa succede se l'intervento richiede più di una visita?}
\seg{L'artigiano comunica con la segretaria. Successivamente la Segretaria concorda telefonicamente con il cliente la data del prossimo intervento.}
\dom{Cosa succede se il cliente richiama dopo alcuni giorni affermando che il lavoro non ha risolto definitivamente il problema?}
\seg{Si concorda un nuovo intervento.}
\dom{Ma il primo intervento viene fatturato?}
\seg{In questo caso, a parte rari casi, il primo intervento si considera come non effettuato. Il materiale per\`o viene addebitato. Se per caso al primo intervento \`e stata sostituita una valvola con un certo costo, e nel secondo intervento la valvola risulta rotta, se questa non \`e sostituibile in garanzia allora viene addebitata.}
\dom{Cosa succede se il cliente non è in casa il giorno dell'appuntamento?}
\seg{Successivamente viene richiamato il cliente e si riprende l'appuntamento. In teoria ci sarebbe da pagare una penale per il mancato intervento. Normalmente a parte con clienti che d'abitudine fanno saltare appuntamenti, questa penale non viene applicata, oppure viene indicata come sconto.}
\dom{Come viene gestita l'assegnazione di mezzi e strumenti aziendali ai vari artigiani?}
\seg{Ogni Artigiano tipicamente ha il suo furgoncino e i suoi attrezzi}
\dom{Come viene gestito il pagamento da parte dei clienti?}
\seg{Tipicamente con bonifico, a volte carta di credito, a volte assegno, a volte contanti}
\dom{Che durata e tipologia ha la garanzia dei vari interventi?}
\seg{Dipende dai componenti}
\dom{Come sceglie l'operaio che deve intervenire?}
\seg{Io, giorno per giorno.}

\dom{La sera precedente analizzo gli interventi da effettuare il
  giorno dopo, e li divido per area geografica attigue. Poi alloco le
  diverse aree ai diversi artigiani. Come criterio generale cerco di
  bilanciare il numero di interventi.}
\dom{Come riceve la conferma dagli artigiani per l'intervento?}
\seg{Alla sera mi ritornano tutte le relazioni degli interventi firmate dai clienti.}
\dom{Come tiene traccia degli appuntamenti?}
\seg{Cartaceo, pi\`u foglio di calcolo (ultimamente)}
\dom{Come prevede la durata di un intervento?}
\seg{Normalmente per ogni tipo d'intervento allochiamo un tempo prefissato. Questo tempo \`e stato calcolato sulla base degli interventi precedenti. In particolare ogni intervento ha preallocate 2h, compreso il tempo per raggiungere il cliente.}
\dom{Se il cliente non è soddisfatto cosa succede?}
\seg{Comunica con la segretaria che cerca di comprendere il motivo. Se questo \`e risolvibile tramite altro intervento allora si cerca di risolverlo, altrimenti il capo dell'azienda si prende in carico il problema.}
\dom{E' possibile rateizzare un pagamento?}
\seg{Dubito, ma bisogna chiedere agli amministrativi.}
\dom{Per quanto tempo tenete traccia dei dati dei clienti?}
\seg{Max due/tre anni dopo l'ultimo intervento questi vengono dimenticati.}
\dom{Fornite un'assistenza post-intervento?}
\seg{No, al massimo si tratta di un altro intervento.}
\dom{Su quali parametri viene calcolato il preventivo?}
\seg{Tempo dell'intervento (1.30 circa) e su eventuali pezzi di ricambio se necessari, e sull'urgenza dell'intervento.}
\dom{Quanto tempo impiega a completare la richiesta di intervento?}
\seg{Non abbiamo una stima precisa, ma \`e il nostro collo di bottiglia. Di solito ci vogliono dai 15 ai 30 minuti a cliente. }
\dom{Viene mantenuto uno storico degli interventi effettuati ? Dopo quanto tempo vengono cancellati ?}
\seg{Tutti i documenti sugli interventi effettuati vengono mantenuti per circa 10 anni nell'archivio.}

\end{enumerate}



\subsection{Domande per \verde{Artigiano}}
\begin{enumerate}
  \dom{Quali sono i tuoi compiti?}
  \art{Ogni sera prima di andare a casa, ritiro la lista
    degli interventi da effettuare il giorno dopo}
  \dom{In cosa consiste un intervento?}
  \art{Recarsi a casa del cliente, valutare il problema,
    eseguire l'intervento se si pu\`o e infine compilare un rapporto
    che il Cliente deve firmare.}
  \dom{Cosa succede se l'intervento ha bisogno di pezzi aggiuntivi?}
  \art{La sera precedente nel ritirare la lista degli
    interventi, controllo se ci sono pezzi che possono servire. Una
    volta individuati chiamo il nostro magazzino di fiducia e chiedo
    di prepararmi la lista dei pezzi di ricambio. La mattina prima di
    recarmi presso i clienti, passo presso il magazzino e ritiro i pezzi.}
  \dom{Cosa succede se la lista preventiva dei pezzi non \`e esaustiva?}
  \art{Mi reco comunque sul posto e valuto la situazione.}
  \dom{A quel punto se il pezzo \`e indispensabile?}
  \art{Allora chiamo la segretaria e rischedulo l'intervento.}
  \dom{E se il pezzo non fosse disponibile a magazzino e ci
    volesse pi\`u tempo per arrivare?}
  \art{Lo segnalerei alla segretaria che si occuperà di rischedulare l'intervento.}
  \dom{Se l'intervento va a buon fine, cosa succede?}
  \art{Compilo una relazione che poi faccio firmare al cliente e riporto alla segretaria}
  \dom{E che dati ci sono nella relazione?}
  \art{La descrizione dettagliata dell'intervento (manodopera), pezzi
    di ricambio usati, dati del cliente, firma cliente, firma
    artigiano, data. }
  \dom{Quanto dura un intervento in media?}
  \art{In media circa due ore}
  \dom{Cosa succede se non finisci in tempo?}
  \art{Fino a 30 minuti oltre il tempo prefissato non succede nulla. Oltre i 30 minuti segnalo in segreteria che generalmente allerta gli artigiani reperibili.}
  \dom{Cosa succede se un cliente ti ruba gli attrezzi? Devi avviare una controversia?}
  \art{Mai successo che un cliente mi rubasse gli attrezzi.}
  \dom{Cosa succede se perdi degli attrezzi?}
  \art{Segnalo all'amministrazione, me li ricompro da magazzini autorizzati dove abbiamo dei conti aperti.}
  \dom{Cosa succede se mentre lavori il cliente ti dice che devi cambiare qualcosa perchè stai sbagliando?}
  \art{Consiglio a lui di fare l'intervento. (ride...)}
  \dom{Come vieni contattato dalla segretaria?}
  \art{Per telefono, o di persona}.
%4) esattamente, in cosa consistono le spese effettuate (manodopera + ricambi)?
\dom{se non hai un pezzo, e la riparazione è urgente, puoi acquistarlo da solo?}
\art{Si}
\dom{cosa succede se non trovi il cliente?}
\art{Comunico con la Segretaria. Generalmente aspetto per circa mezz'ora al massimo un ora e poi passo all'intervento sucessivo lasciando un biglietto di mancato intervento e con tutte le istruzioni in cassetta }
\dom{Cosa succede se arrivato dal cliente ti accorgi che il problema non lo puoi risolvere ma ci vuole un altri tipo di artigiano?}
\art{Comunico alla segretaria che si organizza con il cliente.}
\dom{Si adebbita comunque la visione del problema? Come si rischedula?}
\art{Non saprei con precisione. Credo che per evitare di scontentare il cliente si cerca di risolvere il problema come se fosse urgente.}
\dom{Se fosse richiesto un artigiano aggiuntivo (altro campo di competenze)?}
\art{Di solito per piccoli interventi non accade. In rari casi in cui un intervento necessita di un altro artigiano si riporta alla segretaria che schedula il nuovo intervento.}
\dom{Se l'intervento non va a buon fine cosa succede?}
\art{Io faccio report del perch\`e non si pu\`o effettuare e poi lo comunico alla segretaria che si accorda con il cliente.}
\dom{Quali possono essere i motivi per cui un intervento non va a buon fine?}
\art{Nella mia esperienza sono diversi, i pezzi sono non sono pi\`u disponibili, la riparazione costerebbe pi\`u di cambiare l'intero oggetto dell'intervento, problemi strutturali, lil pezzo da ordinare ci metterebbe troppo tempo per arrivare.....}
\dom{Se la segretaria sbaglia il tempo necessario per eseguire il lavoro, come gestisce una eventuale sovrapposizione?}
\art{Comunico con la Segretaria che cerca di risolvere le sovrapposizioni}
\dom{Cosa deve fare se non riesce a rispettare gli interventi della giornata?}
\art{Comunico con la Segretaria che cerca di riallocare gli interventi.}
\dom{Cosa succede se arriva da un cliente e non deve intervenire?}
\art{Io faccio report e poi il tutto viene gestito dalla Segretaria/Amministrazione}
\dom{Come gestisce una controversia con il cliente?}
\art{Comunico con la segretaria che cerca di risolvere il problema.}
\dom{Come si comporta se al momento dell'intervento manca il materiale/componente corretto?}
\art{Chiamo i negozi pi\`u vicini per capire se hanno a disposizione i pezzi. Se si e se il cliente ha tempo, vado a prendere i pezzi e torno. In questo caso comunico alla segretaria di rischedularmi il prossimo appuntamento. Se il pezzo non \`e disponibile allora chiedo di ordinarlo e poi con il cliente e la segretaria scheduliamo un nuovo appuntamento.}
\dom{Qual'è la procedura che si esegue nel caso di un'emergenza?}
\art{Dal mio punto di vista non ci sono differenze.}
\dom{Nel caso in cui un'urgenza debba sovrapporsi ad un intervento ordinario, come viene rischedulato?}
\art{Bisogna chiedere alla Segretaria, ma nella mia esperienza non ci sono interventi d'urgenza che possano cancellare o far rischedulare interventi ordinari gi\`a schedulati.}
%Ci sono problematiche nella interpretazione dei lavori assegnati ?
%1.2 Ti è mai stato assegnato un compito di non tua competenza ? Se sì, cosa succede ?
%7.1 Cosa succede se l'intervento non va a buon fine?
%9. Cosa succede se l'intervento non viene completato nei tempi previsti?
\dom{Puoi richiedere l'aiuto di un collega? Se si come si registra la sua collaborazione?}
\art{Difficilmente ci\`o pu\`o avvenire. Nel caso di problemi gravi o che non ho mai affrontato chiamo qualche mio collega in via confidenziale e chiedo a loro qualche aiuto. Se qualcuno \`e libero \`e accaduto che passasse a dare un'occhiata.}
\dom{Hai avuto difficoltà nel compilare il rapporto ? }
\art{No}
\dom{Cosa fai quando c'\`e un intervento?}
\art{Analizzo il problema e cerco una soluzione.}
\dom{Quali sono i dati pi\`u importanti nella richesta?}
\art{Le marche e i tipo di pezzi da riparare, eventualmente corredate da foto.}
\end{enumerate}



\subsection{Domande per \rosso{Cliente} \nero{che apre una richiesta}}
\begin{enumerate}
\dom{\'E soddisfatto del servizio?}
\dom{Quali sono i problemi che ha riscontrato?}
\dom{\'E soddisfatto dei tempi del servizio?}
\dom{Vli piacerebbe prenotare un intervento tramite un applicazione?}
\cli{No, preferisco il contatto telefonico di una persona}
\dom{Ci sceglierebbe di nuovo per lavori futuri?}
\cli{Certamente}
\dom{Auale è il metodo di pagamento utilizzabile preferito?}
\cli{Il bonifico bancario parlante.}
\dom{E' soddisfatto dalla tipologia di pagamento?}
\cli{Certamente, il bonifico bancario \`e tracciabile e posso scaricarlo dal 730..}
\dom{Come valuti il rapporto qualità-prezzo?}
\cli{Buono, forse alto, ma la competenza degli artigiani \`e molto alta. Tipicamente risolvono il problema nel giorno successivo alla richiesta.}
\dom{Cosa miglioreresti del servizio?}
\cli{I tempi di attesa per prendere la linea sono estenuanti. Da poco hanno aggiunto un servizio richiema. In pratica schiacciando un pulsante quando trovi occupato vieni registrato e vieni richiamato.
 A volte mi \`e successo che non mi richiamassero e dovessi richiamare io.}
\dom{Sei soddisfatto del tempo impiegato dall'artigiano per arriavare (tempo per arrivare a casa tua) a risolvere il problema?}
\cli{Rispettano sempre gli appuntamenti e sono abbastanza puntuali.}
\dom{Sei soddisfatto del tempo impiegato dall'artigiano per risolvere il problema?}
\cli{Si}
\dom{Sei soddisfatto del tempo che la segretaria ha impiegato a capire quale fosse il problema?}
\cli{In generale abbastanza, solo una volta ha dovuto chiedere un consulto con uno specialista.}
\dom{L'amministrativo, ha impiegato poco tempo per inviare la fattura?}
\cli{Dall'intervento non passa mai pi\`u di una settimana.}
\dom{Sopo mesi dall'intervento hai riscontrato ancora lo stesso problema che avevi prima dell'intervento?}
\cli{No}
\dom{La qualità dei pezzi è buona (non si sono rovinati dopo mesi dall'intervento)?}
\cli{A volte si a volte no, dipende dalle marche dei prodotti.}
\dom{Cosa cambierebbe nel servizio, ai fini di migliorarlo?}
\cli{Ci vorrebbe una segretaria in pi\`u}
\dom{Ha mai riscontrato errori nell'assegnazione degli artigiani in relazione al problema da lei esposto (idraulico, elettricista...) ?}
\cli{No}
\dom{Preferirebbe vedere prima un listino prezzi?}
\cli{Ma, non saprei, il preventivo mi \`e sempre parso pi\`u che sufficiente.}
\dom{Vorrebbe poter scegliere l'artigiano?}
\cli{Per ora non mi sono mai trovato male con nessuno degli artigiani che sono venuti. Per\`o con qualcuno mi sono trovato meglio di altri. Se potessi scegliere si sarebbe meglio, ma solo se questo non influisse con le tempistiche.}
\dom{Vorrebbe poter commentare e leggere recensioni sugli artigiani prima di contattarli?}
\cli{Mi piacerebbe dare un giudizio su alcuni artigiani per poterli rivedere ad esempio.}
\dom{In quale modo preferirebbe avere contatti con la ditta?(telefono, mail,piattaforma web)}
\cli{Telefono cellulare e/o mail}
\dom{Che difficoltà ha avuto nello spiegare il problema?}
\cli{Mai avuto un problema.}
\dom{Ha avuto problemi con il pagamento?}
\cli{No}
\dom{La fattura è arrivata nei giusti tempi?}
\cli{Si, molto velocemente}
\dom{Richiameresti quest'azienda in futuro? perchè?}
\cli{Rapidi e precisi, gli artigiani mi sono sembrati competenti.}
\dom{Se non è soddisfatto del servizio cosa succede?}
\cli{Mai successo. Una volta per errore mi hanno rotto un mobile. Mi hanno mandato un carpentiere successivamente che me lo ha riparato.}
\dom{Se il prezzo è alto cosa fa?}
\cli{Provo a contrattare, ma non funziona. Comunque devo dire che ho notato che il prezzo maggiore dell'intervento di solito \`e dovuto ai pezzi di ricambio e questo non dipende dall'azienda.}
\end{enumerate}




\section{Tecniche di Elicitation: Questionario}
In questa sezione dobbiamo elencare i tipi di questionario che si
vogliono somministrare, indicando principalmente l'obiettivo e le
domande/risposte. In particolare vogliamo perseguire due obiettivi:
(i) capire il grado di alfabetizzazione informatica di tutti gli
stakeholder e (ii) capire il grado di conoscenza dei problemi dei
clienti.  Quindi prepariamo due questionari, ogni questionario deve
avere tra le 5 e le 10 domande, e ogni domanda deve avere esattamente
5 risposte, tra cui non lo so, e altro.

\subsection{Questionario alfabetizzazione informatica}
Il questionario ha come obiettivo quello di capire come i dipendenti
dell'azienda si rapportano con il computer.  \`E diviso in due parti,
nella prima ci sono domande di anagrafica (et\`a, Famiglia, possiedi
pc), e una seconda parte domande a carattere generale sul PC (Qual'\`e
il tuo grado di conoscenza di Office?)
\href{http://www00.unibg.it/dati/bacheca/313/9143.pdf}{Esempio di Questionario}
%% \begin{enumerate}
%% \item
%%   \begin{enumerate}
%%   \item 
%%   \end{enumerate}
%% \end{enumerate}  


\section{Valutazione di applicazioni equivalenti}
In questa sezione descriviamo uno o pi\`u applicazioni che possono
essere d'interesse e avere funzionalit\`a in comune con l'applicazione
che si vuole sviluppare. In ogni sottosezione indicare il link e le
funzionalit\`a spiegandone le caratteristiche.




\end{document}
