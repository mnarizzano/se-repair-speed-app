
\begin{enumerate}
\dom{Quali sono i tuoi compiti?}
\seg{Rispondere al telefono e raccogliere le richieste di intervento.}
\dom{Che cos'è una richiesta d'intervento?}
\seg{\`E un foglio dove sono registrati i dati del cliente e i dati della richiesta.}
\dom{Quali sono i dati del cliente? I dati della richiesta?}
\seg{... e dove \`e descritto il tipo di intervento. ....}
\dom{Come fai a capire il tipo di intervento?}
\seg{Sulla base della mia esperienza e tramite descrizione
  con il cliente, alla fine il tipo di intervento viene registrato
  sulla richiesta e concordato con il cliente.}
\dom{Come si conclude la richiesta di intervento?}
\seg{Devo fare un preventivo e chiedere al cliente se per lui va bene.}
\dom{Il cliente deve rispondere subito?}
\seg{No, dopo una settimana ogni richiesta non accettata scade.}
\dom{Se il cliente accetta cosa succede?}
\seg{Guardo le agende degli artigiani e propongo alcune date al cliente.}
\dom{Entro quanto tempo deve rispondere?}
\seg{Dopo una settimana il preventivo non \`e pi\`u valido, salvo accordi con il cliente.}
\dom{Se il cliente non accetta il preventivo cosa succede?}
\seg{Il preventivo viene cestinato.}
\dom{Se il cliente concorda una data cosa succede?}
\seg{Registro la data sulla richiesta d'intervento, aggiorno le agende degli artigiani.}
\dom{Quali sono i tuoi problemi in queste procedure?}
\seg{Capire il tipo di intervento, fare un preventivo sensato, recuperare i dati dei clienti, la gestione dell'agenda.....}
\dom{Perch\`e il capire il tipo di intervento \`e un problema?}
\seg{Perc\`e non sono ne un idraulico, ne un elettricista e nemmeno un carpentiere. Inoltre i problemi sono molto diversi.}
\dom{E come fai a capire l'intervento?}
\seg{Principalmente tramite l'esperienza. Se non ricordo a
    memoria, vado a scartabellare tra gli interventi passati che sono
    catalogati per tipologia e poi discuto con il cliente.}
\dom{E se nemmeno cos\`i si viene a capo?}
\seg{Attacco, chiamo un esperto e cerco di farmi un idea e poi richiamo il cliente.}
\dom{Parlami della gestione dell'agenda, come viene fatta?}
\seg{O mamma, un vero incubo. Per ogni mese ho un foglio per
  ciascun artigiano dove sono segnati i giorni e nei giorni i suoi
    impegni. Inoltre Per ogni categoria ho un foglio che rappresenta
    il riassunto del mese. Quando devo aggiungere un nuovo intervento
    prima guardo il riassunto e poi i singoli.}
\dom{Ma per chi sono i fogli?}
\seg{I fogli nominativi sono per gli artigiani, i riassunti
    per me per evitare di perdermi. Recentemente ho cominciato a
    registare i fogli di riassunto su un file excel.}
\dom{Come vengono allocati gli interventi? Come si fanno a
    evitare sovrapposizioni? Quanto pu\`o durare un'intervento?}

\seg{Noi assumiamo che per eseguire un'intervento ci
    vogliano circa 2 ore, quindi ne scheduliamo 2 al mattino e due al
    pomeriggio. Come norma per\`o ne scheduliamo tre in modo da
    lasciare spazio per le urgenze.}
\dom{Che differenza c'\`e tra un intervento normale e uno d`urgenza?.}
\seg{Costi e tempi di risposta diversi.}

\dom{Cosa succede se l'intervento non andrà a buon fine}

\dom{Cosa succede se non ci sono artigiani disponibili?}
\seg{Si cambia giorno fino a quando non si trova un artigiano disponibile.}
\dom{Quali sono gli interventi più generali?}
\dom{Quanto spesso non capisci il tipo di intervento?}
\seg{All'inizio era molto frequente, ma con il passare del tempo sono diventata pi\`u esperta. Ci sono giorni che non capita mai, altri che capita una volta. Nell'ultimo anno non ricordo che sia mai capitato pi\`u di una volta al giorno.}
\dom{Cosa succede se non si fa la detection dell'intervento?}
\seg{Scusi pu\`o ripetere?}
\dom{Intendevo: se non riesce ad individuare il tipo di intervento da effettuare, che cosa succede?}
\seg{Ah beh, metto in attesa il cliente, e chiamo l'artigiano che mi sembra possa essere assegnato all'intervento e con lui discuto. Continuo a chiamare fino a che non mi risponde qualcuno che mi cataloga l'intervento come uno degli interventi che pu\`o effettuare la sua categoria.}
\dom{E se anche cos\`i non si riuscisse a individuare il tipo di intervento?}
\seg{Non \`e mai capitato. Immagino per\`o che ci sia bisogno di un sopralluogo.}
\dom{Cosa succede se non ci sono artigiani da assegnare alla richiesta dell'intervento?}
\seg{Si cambia giorno.}
\dom{Se è in corso un intervento normale, come viene gestito uno d'urgenza che viene richiesto nel frattempo e nessuno specializzato è libero?}
\seg{Sono da considerarsi d'urgenza gli interventi effettuati entro le 12 ore dalla chiamata. Tipicamente un artigiano per tipo \`e reperibile ogni giorno per gli interventi d'urgenza.}
\dom{E se arriva una richiesta d'urgenza quando l'artigiano reperibile \`e impegnato?}
\seg{Si mette in coda la richiesta.}
\dom{E se la coda diventa troppo lunga?}
\seg{A questo punto chiamo gli altri artigiani di quel tipo che non sono impegnati e ne cerco uno da aggiungere a quelli reperibili.}
\dom{E se non ne trova?}
\seg{Metto le richieste in coda.}
\dom{Cosa accade se è richiesto un intervento di urgenza, ma è in corso lo stato di allerta meteo?}
\seg{Solo in caso di allerta rossa rischeduliamo tutti gli appuntamenti non urgenti chiamando il cliente.}
\dom{Usi programmi come office?}
\seg{Si}
\dom{Qual'è il tuo livello di utilizzo?}
\seg{Riesco a utilizzare bene word ed excel con multitabelle. Riesco anche a inserire formule nelle celle per eseguire dei calcoli.}
\dom{In base a cosa scegli l'artigiano?}
\seg{In base al tipo d'intervento, e in base ai giorni richiesti dal cliente.}
\dom{Come decidi la priorità per ogni riparazione?}
\seg{Il primo che arriva sceglie.}
\dom{Se un artigiano fa male un intervento, e il cliente si lamenta, come gestisci la situazione?}
\seg{Avviso il padrone che si fa carico di controllare il lavoro.}
\dom{Se un artigiano crea un danno al cliente, come viene gestista la situazione?}
\seg{Se il danno viene riconosciuto dall'artigiano allora ci occupiamo di risarcire il cliente tramite un'assicurazione che abiamo stipulato.}
\dom{Si può modificare data, artigiano o altri campi dell'intervento? Esiste già nel caso una procedura a riguardo?}
\seg{Si pu\`o modificare la data dell'intervento fino a 12 ore prima dell'intervento stesso. Il cliente deve chiamare in segreteria.}
\dom{Come viene comunicato un intervento urgente?}
\seg{L'urgenza la stabilisce il cliente, se vuole che l'intervento venga eseguito entro la giornata (entro 6/8 ore) allora deve comunicarlo a me mentre siamo al telefono.}
\dom{Ci sono già domande precompilate per l'individuazione del tipo di intervento?}
\seg{Si}
\dom{Come si riallocano gli interventi in caso di problemi?}
\seg{Se il problema avviene a seguito del primo intervento, con il cliente si concorda un'altra data in cui finire l'intervento. Per la riallocazione bisogna sentire le parti interessate (cliente, magazzino per eventuali pezzi e artigiano).}
\dom{E se un artigiano non pu\`o eseguire l'intervento? Ad esempio perch\`e un altro intervento ha richiesto pi\`u tempo, oppure malato, oppure incidente, oppure traffico (hanno chiuso le strade per arricarci)?}
\seg{Se il cliente \`e raggiungibile si cerca di allocare l'intervento gli altri artigiani di quel tipo che lavorano in quella giornata e siano parzialmente scarichi. Se tutti gli artigiani sono a tappo e se non ci sono troppe urgenze si utilizzano gli artigiani reperibili, altrimenti si chiama il cliente e si concorda un altro giorno.}
\dom{Qual'è la procedura da seguire se il cliente richiama per modificare la data dell'intervento?}
\seg{Se fatta entro le 12 ore precedenti l'intervento semplicemente si cancella l'appuntamento e si procede a concordare con il cliente una nuova data.}
\dom{Ma questo non influenza anche le agende degli altri artigiani dello stesso tipo?}
\seg{Non necessariamente. L'agenda viene creata la sera precedente in base agli interventi e le aree geografiche. Per cui se una persona chiama il giorno prima le agende non sono ancora pronte.}
\dom{In una richiesta d'intervento pu\`o essere cambiato anche il tipo d'intervento? Se il cliente si accorgesse di aver sbagliato a descrivere il problema?}
\seg{Se fatta entro le 12 ore precedenti l'intervento semplicemente si cancella l'appuntamento e si procede a concordare con il cliente una nuova data.}
\dom{Cosa succede se l'intervento richiede più di una visita?}
\seg{L'artigiano comunica con la segretaria. Successivamente la Segretaria concorda telefonicamente con il cliente la data del prossimo intervento.}
\dom{Cosa succede se il cliente richiama dopo alcuni giorni affermando che il lavoro non ha risolto definitivamente il problema?}
\seg{Si concorda un nuovo intervento.}
\dom{Ma il primo intervento viene fatturato?}
\seg{In questo caso, a parte rari casi, il primo intervento si considera come non effettuato. Il materiale per\`o viene addebitato. Se per caso al primo intervento \`e stata sostituita una valvola con un certo costo, e nel secondo intervento la valvola risulta rotta, se questa non \`e sostituibile in garanzia allora viene addebitata.}
\dom{Cosa succede se il cliente non è in casa il giorno dell'appuntamento?}
\seg{Successivamente viene richiamato il cliente e si riprende l'appuntamento. In teoria ci sarebbe da pagare una penale per il mancato intervento. Normalmente a parte con clienti che d'abitudine fanno saltare appuntamenti, questa penale non viene applicata, oppure viene indicata come sconto.}
\dom{Come viene gestita l'assegnazione di mezzi e strumenti aziendali ai vari artigiani?}
\seg{Ogni Artigiano tipicamente ha il suo furgoncino e i suoi attrezzi}
\dom{Come viene gestito il pagamento da parte dei clienti?}
\seg{Tipicamente con bonifico, a volte carta di credito, a volte assegno, a volte contanti}
\dom{Che durata e tipologia ha la garanzia dei vari interventi?}
\seg{Dipende dai componenti}
\dom{Come sceglie l'operaio che deve intervenire?}
\seg{Io, giorno per giorno.}

\dom{La sera precedente analizzo gli interventi da effettuare il
  giorno dopo, e li divido per area geografica attigue. Poi alloco le
  diverse aree ai diversi artigiani. Come criterio generale cerco di
  bilanciare il numero di interventi.}
\dom{Come riceve la conferma dagli artigiani per l'intervento?}
\seg{Alla sera mi ritornano tutte le relazioni degli interventi firmate dai clienti.}
\dom{Come tiene traccia degli appuntamenti?}
\seg{Cartaceo, pi\`u foglio di calcolo (ultimamente)}
\dom{Come prevede la durata di un intervento?}
\seg{Normalmente per ogni tipo d'intervento allochiamo un tempo prefissato. Questo tempo \`e stato calcolato sulla base degli interventi precedenti. In particolare ogni intervento ha preallocate 2h, compreso il tempo per raggiungere il cliente.}
\dom{Se il cliente non è soddisfatto cosa succede?}
\seg{Comunica con la segretaria che cerca di comprendere il motivo. Se questo \`e risolvibile tramite altro intervento allora si cerca di risolverlo, altrimenti il capo dell'azienda si prende in carico il problema.}
\dom{E' possibile rateizzare un pagamento?}
\seg{Dubito, ma bisogna chiedere agli amministrativi.}
\dom{Per quanto tempo tenete traccia dei dati dei clienti?}
\seg{Max due/tre anni dopo l'ultimo intervento questi vengono dimenticati.}
\dom{Fornite un'assistenza post-intervento?}
\seg{No, al massimo si tratta di un altro intervento.}
\dom{Su quali parametri viene calcolato il preventivo?}
\seg{Tempo dell'intervento (1.30 circa) e su eventuali pezzi di ricambio se necessari, e sull'urgenza dell'intervento.}
\dom{Quanto tempo impiega a completare la richiesta di intervento?}
\seg{Non abbiamo una stima precisa, ma \`e il nostro collo di bottiglia. Di solito ci vogliono dai 15 ai 30 minuti a cliente. }
\dom{Viene mantenuto uno storico degli interventi effettuati ? Dopo quanto tempo vengono cancellati ?}
\seg{Tutti i documenti sugli interventi effettuati vengono mantenuti per circa 10 anni nell'archivio.}

\end{enumerate}
