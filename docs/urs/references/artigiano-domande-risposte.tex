\begin{enumerate}
  \dom{Quali sono i tuoi compiti?}
  \art{Ogni sera prima di andare a casa, ritiro la lista
    degli interventi da effettuare il giorno dopo}
  \dom{In cosa consiste un intervento?}
  \art{Recarsi a casa del cliente, valutare il problema,
    eseguire l'intervento se si pu\`o e infine compilare un rapporto
    che il Cliente deve firmare.}
  \dom{Cosa succede se l'intervento ha bisogno di pezzi aggiuntivi?}
  \art{La sera precedente nel ritirare la lista degli
    interventi, controllo se ci sono pezzi che possono servire. Una
    volta individuati chiamo il nostro magazzino di fiducia e chiedo
    di prepararmi la lista dei pezzi di ricambio. La mattina prima di
    recarmi presso i clienti, passo presso il magazzino e ritiro i pezzi.}
  \dom{Cosa succede se la lista preventiva dei pezzi non \`e esaustiva?}
  \art{Mi reco comunque sul posto e valuto la situazione.}
  \dom{A quel punto se il pezzo \`e indispensabile?}
  \art{Allora chiamo la segretaria e rischedulo l'intervento.}
  \dom{E se il pezzo non fosse disponibile a magazzino e ci
    volesse pi\`u tempo per arrivare?}
  \art{Lo segnalerei alla segretaria che si occuperà di rischedulare l'intervento.}
  \dom{Se l'intervento va a buon fine, cosa succede?}
  \art{Compilo una relazione che poi faccio firmare al cliente e riporto alla segretaria}
  \dom{E che dati ci sono nella relazione?}
  \art{La descrizione dettagliata dell'intervento (manodopera), pezzi
    di ricambio usati, dati del cliente, firma cliente, firma
    artigiano, data. }
  \dom{Quanto dura un intervento in media?}
  \art{In media circa due ore}
  \dom{Cosa succede se non finisci in tempo?}
  \art{Fino a 30 minuti oltre il tempo prefissato non succede nulla. Oltre i 30 minuti segnalo in segreteria che generalmente allerta gli artigiani reperibili.}
  \dom{Cosa succede se un cliente ti ruba gli attrezzi? Devi avviare una controversia?}
  \art{Mai successo che un cliente mi rubasse gli attrezzi.}
  \dom{Cosa succede se perdi degli attrezzi?}
  \art{Segnalo all'amministrazione, me li ricompro da magazzini autorizzati dove abbiamo dei conti aperti.}
  \dom{Cosa succede se mentre lavori il cliente ti dice che devi cambiare qualcosa perchè stai sbagliando?}
  \art{Consiglio a lui di fare l'intervento. (ride...)}
  \dom{Come vieni contattato dalla segretaria?}
  \art{Per telefono, o di persona}.
%4) esattamente, in cosa consistono le spese effettuate (manodopera + ricambi)?
\dom{se non hai un pezzo, e la riparazione è urgente, puoi acquistarlo da solo?}
\art{Si}
\dom{cosa succede se non trovi il cliente?}
\art{Comunico con la Segretaria. Generalmente aspetto per circa mezz'ora al massimo un ora e poi passo all'intervento sucessivo lasciando un biglietto di mancato intervento e con tutte le istruzioni in cassetta }
\dom{Cosa succede se arrivato dal cliente ti accorgi che il problema non lo puoi risolvere ma ci vuole un altri tipo di artigiano?}
\art{Comunico alla segretaria che si organizza con il cliente.}
\dom{Si adebbita comunque la visione del problema? Come si rischedula?}
\art{Non saprei con precisione. Credo che per evitare di scontentare il cliente si cerca di risolvere il problema come se fosse urgente.}
\dom{Se fosse richiesto un artigiano aggiuntivo (altro campo di competenze)?}
\art{Di solito per piccoli interventi non accade. In rari casi in cui un intervento necessita di un altro artigiano si riporta alla segretaria che schedula il nuovo intervento.}
\dom{Se l'intervento non va a buon fine cosa succede?}
\art{Io faccio report del perch\`e non si pu\`o effettuare e poi lo comunico alla segretaria che si accorda con il cliente.}
\dom{Quali possono essere i motivi per cui un intervento non va a buon fine?}
\art{Nella mia esperienza sono diversi, i pezzi sono non sono pi\`u disponibili, la riparazione costerebbe pi\`u di cambiare l'intero oggetto dell'intervento, problemi strutturali, lil pezzo da ordinare ci metterebbe troppo tempo per arrivare.....}
\dom{Se la segretaria sbaglia il tempo necessario per eseguire il lavoro, come gestisce una eventuale sovrapposizione?}
\art{Comunico con la Segretaria che cerca di risolvere le sovrapposizioni}
\dom{Cosa deve fare se non riesce a rispettare gli interventi della giornata?}
\art{Comunico con la Segretaria che cerca di riallocare gli interventi.}
\dom{Cosa succede se arriva da un cliente e non deve intervenire?}
\art{Io faccio report e poi il tutto viene gestito dalla Segretaria/Amministrazione}
\dom{Come gestisce una controversia con il cliente?}
\art{Comunico con la segretaria che cerca di risolvere il problema.}
\dom{Come si comporta se al momento dell'intervento manca il materiale/componente corretto?}
\art{Chiamo i negozi pi\`u vicini per capire se hanno a disposizione i pezzi. Se si e se il cliente ha tempo, vado a prendere i pezzi e torno. In questo caso comunico alla segretaria di rischedularmi il prossimo appuntamento. Se il pezzo non \`e disponibile allora chiedo di ordinarlo e poi con il cliente e la segretaria scheduliamo un nuovo appuntamento.}
\dom{Qual'è la procedura che si esegue nel caso di un'emergenza?}
\art{Dal mio punto di vista non ci sono differenze.}
\dom{Nel caso in cui un'urgenza debba sovrapporsi ad un intervento ordinario, come viene rischedulato?}
\art{Bisogna chiedere alla Segretaria, ma nella mia esperienza non ci sono interventi d'urgenza che possano cancellare o far rischedulare interventi ordinari gi\`a schedulati.}
%Ci sono problematiche nella interpretazione dei lavori assegnati ?
%1.2 Ti è mai stato assegnato un compito di non tua competenza ? Se sì, cosa succede ?
%7.1 Cosa succede se l'intervento non va a buon fine?
%9. Cosa succede se l'intervento non viene completato nei tempi previsti?
\dom{Puoi richiedere l'aiuto di un collega? Se si come si registra la sua collaborazione?}
\art{Difficilmente ci\`o pu\`o avvenire. Nel caso di problemi gravi o che non ho mai affrontato chiamo qualche mio collega in via confidenziale e chiedo a loro qualche aiuto. Se qualcuno \`e libero \`e accaduto che passasse a dare un'occhiata.}
\dom{Hai avuto difficoltà nel compilare il rapporto ? }
\art{No}
\dom{Cosa fai quando c'\`e un intervento?}
\art{Analizzo il problema e cerco una soluzione.}
\dom{Quali sono i dati pi\`u importanti nella richesta?}
\art{Le marche e i tipo di pezzi da riparare, eventualmente corredate da foto.}
\end{enumerate}

