\begin{enumerate}
  \dom{Quali sono i dati utilizzati per la fattura?}
  \amm{}
  \dom{A quando risalgono le ultime fatture conservate?}
  \amm{Questa settimana}
  \dom{Quali sono i metodi di pagamento disponibili al cliente?}
  \amm{Bonifico, Contanti, Carta di Credito (on-site), Bancomat(on-site) assegno}
  \dom{Come viene monitorato il pagamento del cliente? dopo quanto vi è il sollecito?}
  
  \dom{Tutte le fatture da pagare sono tenute in un
    raccoglitore. Periodicamente controllo sul conto dell'azienda se
    sono state pagate. Se sono state pagate le archivio, altrimenti faccio un sollecito.}

  \dom{Dopo quanti giorni di mancato pagamento viene inviata la prima sollecitazione?}
  \amm{Tipicamente 30 giorni.}
  \dom{Ci sono differenze di fatturazione fra interventi normali e interventi d'urgenza? }
  \amm{No, stesso tipo di fattura. Cambiano i prezzi.}
  \dom{Se il cliente rifiuta il pagamento perché è tropo caro cosa succede?}

  \amm{Difficile perch\`e preventivato in anticipo e accettato. Se
    per\`o un pezzo secondo il cliente costa troppo, o lo procura lui
    oppure non eseguiamo l'intervento. Nel caso comunque dobbiamo addebitare la chiamata.}
  \dom{Come prepara la fattura?}
  \amm{Al computer ho dei template dove inserisco i dati del cliente e
    i dati della relazione dell'artigiano.}
  \dom{Come ci si comporta se un cliente non paga?}

  \amm{Prima si parte con solleciti soft (mail, o messaggi), poi
    chiamate per capire se ci sono problemi, ed infine solleciti
    sempre pi\`u insistenti fino alla scrittura da parte di un
    avvocato. }
  \dom{Come sono gestiti i pagamenti dei dipendenti?}
  \amm{Cedolini mensili e conguaglio a fine anno.}

  \dom{Quali sono le problematiche che riscontri maggiormente?}

  \amm{Non tante, ma inconsistenza dei dati (Codice Fiscale sbagliato
    ad esempio), a volte incongruenza delle relazioni. Il pi\`u grosso
    problema riscontrato \`e il prezzo dei componenti. A fine mese le
    ditte da cui acquistiamo ci mandano il conto con tutti i pezzi
    acquistati. Il mio compito \`e controllare che i prezzi tra quelli
    fatturati e quelli che noi abbiamo fatturato ai clienti. Spesso i
    prezzi non combaciano e rischiamo di fatturare un pezzo ad un
    cliente ad un costo inferiore a quello a cui noi lo abbiamo
    pagato. Un lavoro veramente difficile.}
  \dom{Quali sono i tuoi compiti?}
  
  \amm{Mi occupo degli aspetti amministrativi, da preparare le
    fatture, a spedirle, a controllare i pagamenti fino a controllare
    che i prezzi che ci fatturano combaciano con i preventivi che ci
    avevano fatto.}
 
  \dom{Che difficoltà hai trovato nella compilazione della fattura?}
  
  \amm{Ripetizione dei dati dei clienti, che non essendo registrati su
    nessun database devono essere ricopiati ogni volta.}
  
  \dom{Ci sono problemi di passaggio della relazione di avvenuta riparazione tra la segreteria e il suo ufficio?}
  
  \amm{Non che io ricordi. A volte si perdono i fogli, oppure ci si
    dimentica di archiviarli quindi potrebbe diventare difficile
    recuperarli.}

  \dom{Come rilasciate le fatture?}

  \amm{Formato elettronico e formato cartaceo. Dal 2019 \`e entrato in
    vigore la fatturazione elettronica.}
  
  \dom{Gli artigiani compilano sempre in modo adeguato i rapporti?}
  \amm{Quasi sempre}
  
  \dom{Nel caso di mancate informazioni come vi comportate?}
  \amm{Contattiamo la segretaria, artigiano e infine cliente.}
  \dom{Come gestite l'acquisto di nuovi pezzi, strumenti ecc...?}
  
  \amm{Abbiamo dei conti aperti con alcuni negozi dove i nostri
    artigiani ordinano i pezzi. A fine mese i negozi ci mandano la
    lista dei pezzi acquistati e il totale di quanto dobbiamo pagare.}

  \dom{In caso un pezzo non fosse disponibile presso i soliti
    fornitori o fosse disponibile da altri negozi in tempi/prezzi
    minori cosa di fa?}

  \amm{Nel caso l'artigiano anticipa i soldi e compera il pezzo. Lo
    scontrino viene poi rimborsato all'artigiano. In genere il prezzo
    non \`e un problema, a meno che non costi tanto di meno, quello
    che conta sono le tempistiche. Per cui un artigiano \`e
    autorizzato a comperare un pezzo da un negozio esterno solo se
    disponibile in tempi minori.}

  \dom{Ogni quanto vengono consegnate le relazioni riguardanti gli interventi?}
  \amm{Ogni giorno}
  \dom{Che tipologia di tecnologie informative sta attualmente utilizzando nel suo lavoro?}
  \amm{Word per compilare le fatture e un programma per mandare le mail}
\end{enumerate}
