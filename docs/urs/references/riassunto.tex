\documentclass[11pt]{article}
\topmargin -.5in
\textheight 9in
\oddsidemargin -.25in
\evensidemargin -.25in
\textwidth 7in
\usepackage{xcolor}
\usepackage{hyperref}
\newcommand{\rosso}[1]{\color[rgb]{1,0,0}{#1}}
\newcommand{\blu}[1]{\color[rgb]{0,0,1}{#1}}
\newcommand{\bianco}[1]{\color[rgb]{1,1,1}{#1}}
\newcommand{\verde}[1]{\color[rgb]{0,0.65,0.31}{#1}}
\newcommand{\giallo}[1]{\color[rgb]{0.96,0.67,0}{#1}}
\newcommand{\nero}[1]{\color[rgb]{0,0,0}{#1}}



\begin{document}

% ========== Edit your name here
\author{}
\title{Report di un primo briefing sul progetto Repair Speed-App}
\maketitle

\section{Contesto}
Una piccola ditta esegue interventi di riparazione per la casa:
piccoli interventi di manutenzione elettrica, piccoli interventi di
idraulica e carpenteria (riparazione di mobili, pareti in
legno/cartongesso). Ha diversi dipendenti: Una segretaria, un
amministrativo, circa 30 artigiani suddivisi in idraulici,
elettricisti e carpentieri.
%%
Il flusso di lavoro della ditta \`e il seguente: Un cliente chiama la
segretaria e spiega il problema. La segretaria innanzi tutto apre
una nuova richiesta di intervento (un foglio prestampato), registra
i dati del cliente, parlando con il cliente individua il problema, e
sceglie il tipo di artigiano che si pu\`o occupare del problema, e
infine individua delle possibili date per eseguire il lavoro.  Se il
cliente accetta, sceglie il giorno e l'ora e la segretaria finalizza
il foglio. Chiusa la telefonata registra l'intervento da effettuare.

%%
L'artigiano ogni mattina prende il foglio dove sono descritti gli
interventi che deve effettuare e si reca a domicilio per il risolvere
il problema.
%%
Al termine di ogni intervento l'artigiano deve compilare un rapperto
descrivendo l'intervento e aggiungendo le spese effettuate e il tempo
necessario  a risolvere il problema.
%%
Ottenuta la relazione attestante l'avvenuta riparazione, un
amministrativo compila la fattura con i dati del cliente e la
spedisce, monitorando il pagamento e laddove necessario sollecitando
il pagamento.
%%

 Il cliente, il proprietario della ditta, sostiene che la gestione
 cartacea \`e molto onerosa e macchinosa. Vorrebbe un sistema
 informatico che snellisse le procedure eliminando (quasi)
 completamente la carta.

\section{Concetti Generali}

\subsection{Stakeholders}:
In questa sezione vengono descritti gli stakeholders, chi sono e quali
sono gli obiettivi.

\begin{itemize}
\item Cliente (Padrone Azienda),
\item Segretaria,
\item Amministrativo,
\item Artigiano(Elettricista, Idraulico, Falegname),
\item Cliente.
\end{itemize}

\subsection{Campioni}
In questa sottosezione inseriamo i dati dei campioni da intervistare.
Il Proprietario, La Segretaria, L'Amministrativo,
 L'Idraulico, un cliente vecchio.....

 \subsection{Tecniche}
In questa sezione descriviamo le tecniche che si adotteranno. Le
tecniche verranno poi approfondite nelle prossime sezioni.
\begin{itemize}
\item Interviste mirate a campioni di ogni stakeholder.
\item Questionario Anonimo, a risposta multipla, con poche domande e
  al pi\`u 5 risposte. Non pi\`u di  prodotti concorrenti
\end{itemize}  
%% \subsection{Requisiti iniziali}
%% \begin{itemize}
%%   \item[] La segretaria deve Compilare un \textsl{Preventivo per una Richiesta di Intervento}
%%   \item[] La segretaria deve Compilare una \textsl{Richiesta di Intervento} 
%%   \item[] La segretaria deve Gestire le agende degli artigiani
%%   \item[] L'artigiano deve poter gestire la sua agenda
%%   \item[] L'artigiano deve poter chiudere una Richiesta di intervento
%%   \item[] L'artigiano deve poter compilare  un rapporto per ogni intervento effettuato
%%   \item[] .....
%%   \item[]   
%%   \end{itemize}

\section{Tecniche di Elicitation: Interviste}
In questa sezione elenchiamo alcune domande che vorremmo sottoporre ai
vari stakeholder. Sono divise in sezioni, ed ogni sezione racchiude le
domande per un gruppo di persone (stakeholder).

\subsection{Domande per \blu{Segretaria}}
\begin{enumerate}
\item \nero{Quali sono i tuoi compiti?}
\item \nero{Che cos'è una richiesta d'intervento?}
\item \nero{Quali sono i dati del cliente? I dati della richiesta?}
\item \nero{Come fai a capire il tipo di intervento?}
\item \nero{Come si conclude la richiesta di intervento?}
\item \nero{Il cliente deve rispondere subito?}
\item \nero{Se il cliente accetta cosa succede?}
\item \nero{Se il cliente concorda una data cosa succede?}
\item \nero{Esiste un tempo minimo e uno massimo di intervento?}
\item \nero{Quali sono i tuoi problemi in queste procedure?}
\item \nero{Perch\`e il capire il tipo di intervento \`e un problema?}
\item \nero{Parlami della gestione dell'agenda, come viene fatta?}
\item \nero{Ma per chi sono i fogli dell'agenda?}
\item \nero{Come vengono allocati gli interventi? Come si fanno a
  evitare sovrapposizioni? Quanto pu\`o durare un'intervento?}
\item \nero{Che differenza c'\`e tra un intervento normale e uno d`urgenza?.}
\item \nero{...}
\end{enumerate}



\subsection{Domande per \verde{Artigiano}}
\begin{enumerate}
\item \nero{Quali sono i tuoi compiti?}
\item \nero{In cosa consiste un intervento?}
\item \nero{Cosa succede se l'intervento ha bisogno di pezzi aggiuntivi?}
\item \nero{Cosa succede se la lista preventiva dei pezzi non \`e esaustiva?}
\item \nero{A quel punto se il pezzo \`e indispensabile?}
\item \nero{E se il pezzo non fosse disponibile a magazzino e ci
  volesse pi\`u tempo per arrivare?}
\item \nero{Se l'intervento va a buon fine, cosa succede?}
\item \nero{E che dati ci sono nella relazione?}
\item \nero{...}
\end{enumerate}


\subsection{Domande per \rosso{Cliente} \nero{che apre una richiesta}}
\begin{enumerate}
\item \nero{\'E soddisfatto del servizio?}
\item \nero{Quali sono i problemi che ha riscontrato?}
\item \nero{\'E soddisfatto dei tempi del servizio?}
\item \nero{...}
\end{enumerate}



\section{Tecniche di Elicitation: Questionario}
In questa sezione dobbiamo elencare i tipi di questionario che si
vogliono somministrare, indicando principalmente l'obiettivo e le
domande/risposte. In particolare vogliamo perseguire due obiettivi:
(i) capire il grado di alfabetizzazione informatica di tutti gli
stakeholder e (ii) capire il grado di conoscenza dei problemi dei
clienti.  Quindi prepariamo due questionari, ogni questionario deve
avere tra le 5 e le 10 domande, e ogni domanda deve avere esattamente
5 risposte, tra cui non lo so, e altro.

\subsection{Questionario alfabetizzazione informatica}
Il questionario ha come obiettivo quello di capire come i dipendenti
dell'azienda si rapportano con il computer.  \`E diviso in due parti,
nella prima ci sono domande di anagrafica (et\`a, Famiglia, possiedi
pc), e una seconda parte domande a carattere generale sul PC (Qual'\`e
il tuo grado di conoscenza di Office?)
\href{http://www00.unibg.it/dati/bacheca/313/9143.pdf}{Esempio di Questionario}
%% \begin{enumerate}
%% \item
%%   \begin{enumerate}
%%   \item 
%%   \end{enumerate}
%% \end{enumerate}  


\section{Valutazione di applicazioni equivalenti}
In questa sezione descriviamo uno o pi\`u applicazioni che possono
essere d'interesse e avere funzionalit\`a in comune con l'applicazione
che si vuole sviluppare. In ogni sottosezione indicare il link e le
funzionalit\`a spiegandone le caratteristiche.




\end{document}



Esercizion:

1) Leggere e scaricare il primo report dal repository mnarizzano/repair-speed-app
2) Completare il documento aggiungendo un file di testo contenente:
(i) domande per gli stakeholder (completare quelli che ci sono e aggiungere quelle che ci sono)
(ii) Completare i questionari
(iii) Trovare un'applicazione inerente al progetto e analizzarne le funzionalit\`a
