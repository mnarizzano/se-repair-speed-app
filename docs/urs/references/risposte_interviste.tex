\begin{itemize}
\item[] \rosso{Quali sono i tuoi compiti?}
\item[] \blu{Rispondere al telefono e raccogliere le richieste di intervento.}
\item[] \rosso{Che cos'è una richiesta d'intervento?}
\item[] \blu{\`E un foglio dove sono registrati i dati del cliente e i dati della richiesta.}
\item[] \rosso{Quali sono i dati del cliente? I dati della richiesta?}
\item[] \blu{... e dove \`e descritto il tipo di intervento. ....}
\item[] \rosso{Come fai a capire il tipo di intervento?}
\item[] \blu{Sulla base della mia esperienza e tramite descrizione
  con il cliente, alla fine il tipo di intervento viene registrato
  sulla richiesta e concordato con il cliente.}
\item[] \rosso{Come si conclude la richiesta di intervento?}
\item[] \blu{Devo fare un preventivo e chiedere al cliente se per lui va bene.}
\item[] \rosso{Il cliente deve rispondere subito?}
\item[] \blu{No, dopo una settimana ogni richiesta non accettata scade.}
\item[] \rosso{Se il cliente accetta cosa succede?}
\item[] \blu{Guardo le agende degli artigiani e propongo alcune date al cliente.}
\item[] \rosso{Se il cliente concorda una data cosa succede?}
\item[] \blu{Registro la data sulla richiesta d'intervento, aggiorno le agende degli artigiani.}
\item[] \rosso{Esiste un tempo minimo e uno massimo di intervento?}
  \item[] \rosso{Quali sono i tuoi problemi in queste procedure?}
  \item[] \blu{Capire il tipo di intervento, fare un preventivo sensato, recuperare i dati dei clienti, la gestione dell'agenda.....}
  \item[] \rosso{Perch\`e il capire il tipo di intervento \`e un problema?}
  \item[] \blu{Perc\`e non sono ne un idraulico, ne un elettricista e nemmeno un carpentiere. Inoltre i problemi sono molto diversi.}
  \item[] \rosso{E come fai a capire l'intervento?}
  \item[] \blu{Principalmente tramite l'esperienza. Se non ricordo a
    memoria, vado a scartabellare tra gli interventi passati che sono
    catalogati per tipologia e poi discuto con il cliente.}
  \item[] \rosso{E se nemmeno cos\`i si viene a capo?}
  \item[] \blu{Attacco, chiamo un esperto e cerco di farmi un idea e poi richiamo il cliente.}
  \item[] \rosso{Parlami della gestione dell'agenda, come viene fatta?}
  \item[] \blu{O mamma, un vero incubo. Per ogni mese ho un foglio per
    ciascun artigiano dove sono segnati i giorni e nei giorni i suoi
    impegni. Inoltre Per ogni categoria ho un foglio che rappresenta
    il riassunto del mese. Quando devo aggiungere un nuovo intervento
    prima guardo il riassunto e poi i singoli.}
  \item[] \rosso{Ma per chi sono i fogli?}
  \item[] \blu{I fogli nominativi sono per gli artigiani, i riassunti
    per me per evitare di perdermi. Recentemente ho cominciato a
    registare i fogli di riassunto su un file excel.}
  \item[] \rosso{Come vengono allocati gli interventi? Come si fanno a
    evitare sovrapposizioni? Quanto pu\`o durare un'intervento?}

  \item[] \blu{Noi assumiamo che per eseguire un'intervento ci
    vogliano circa 2 ore, quindi ne scheduliamo 2 al mattino e due al
    pomeriggio. Come norma per\`o ne scheduliamo tre in modo da
    lasciare spazio per le urgenze.}
  \item[] \rosso{Che differenza c'\`e tra un intervento normale e uno d`urgenza?.}
  \item[] \blu{Costi e tempi di risposta diversi.}
    
\end{itemize}



\subsection{\verde{Artigiano}}
  \begin{itemize}
  \item[] \rosso{Quali sono i tuoi compiti?}
  \item[] \verde{Ogni sera prima di andare a casa, ritiro la lista
    degli interventi da effettuare il giorno dopo}
  \item[] \rosso{In cosa consiste un intervento?}
  \item[] \verde{Recarsi a casa del cliente, valutare il problema,
    eseguire l'intervento se si pu\`o e infine compilare un rapporto
    che il Cliente deve firmare.}
  \item[] \rosso{Cosa succede se l'intervento ha bisogno di pezzi aggiuntivi?}
  \item[] \verde{La sera precedente nel ritirare la lista degli
    interventi, controllo se ci sono pezzi che possono servire. Una
    volta individuati chiamo il nostro magazzino di fiducia e chiedo
    di prepararmi la lista dei pezzi di ricambio. La mattina prima di
    recarmi presso i clienti, passo presso il magazzino e ritiro i pezzi.}
  \item[] \rosso{Cosa succede se la lista preventiva dei pezzi non \`e esaustiva?}
  \item[] \verde{Mi reco comunque sul posto e valuto la situazione.}
  \item[] \rosso{A quel punto se il pezzo \`e indispensabile?}
  \item[] \verde{Allora chiamo la segretaria e rischedulo l'intervento.}
  \item[] \rosso{E se il pezzo non fosse disponibile a magazzino e ci
    volesse pi\`u tempo per arrivare?}
  \item[] \verde{Lo segnalerei alla segretaria che si occuperà di rischedulare l'intervento.}
  \item[] \rosso{Se l'intervento va a buon fine, cosa succede?}
  \item[] \verde{Compilo una relazione che poi faccio firmare al cliente e riporto alla segretaria}
  \item[] \rosso{E che dati ci sono nella relazione?}
  \item[] \verde{}
   \end{itemize}
\end{document}
