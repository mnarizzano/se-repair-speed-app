\documentclass[11pt]{article}
\topmargin -.5in
\textheight 9in
\oddsidemargin -.25in
\evensidemargin -.25in
\textwidth 7in
\usepackage{xcolor}
\usepackage{hyperref}
\newcommand{\rosso}[1]{\color[rgb]{1,0,0}{#1}}
\newcommand{\blu}[1]{\color[rgb]{0,0,1}{#1}}
\newcommand{\bianco}[1]{\color[rgb]{1,1,1}{#1}}
\newcommand{\verde}[1]{\color[rgb]{0,0.65,0.31}{#1}}
\newcommand{\giallo}[1]{\color[rgb]{0.96,0.67,0}{#1}}
\newcommand{\nero}[1]{\color[rgb]{0,0,0}{#1}}

\newcommand{\dom}[1]{\item \color[rgb]{0,0,0}{#1}}
\newcommand{\seg}[1]{\item[] \color[rgb]{0,0,1}{#1}}
\newcommand{\art}[1]{\item[] \color[rgb]{0,0.65,0.31}{#1}}
\newcommand{\cli}[1]{\item[] \color[rgb]{1,0,0}{#1}}
\newcommand{\amm}[1]{\item[] \color[rgb]{0.96,0.67,0}{#1}}
\newcommand{\capo}[1]{\item[] \color[rgb]{0,1,0}{#1}}

\begin{document}

% ========== Edit your name here
\author{}
\title{Analisi di Prodotti simili}
\maketitle

\section{\href{https://www.searshomeservices.com/}{Home Maintenance Services}}

L’applicazione elimina quasi del tutto la parte front-end della
segreteria: il cliente sceglie il tipo di intervento, la data,
l’orario ecc.. Immaginiamo ci sia comunque un ufficio di segreteria
atto ad organizzare i vari appuntamenti ed informare i vari
professionisti (oppure un sistema automatico).  Sono presenti comunque
una chat e un numero di telefono nel caso in cui un cliente incontri
delle difficoltà all’interno dell’applicazione, oppure non sappia
descrivere autonomamente la richiesta di intervento.

L’applicazione possiede un sistema di recensioni che può portare vari vantaggi:

\begin{enumerate}
  \item il cliente può esprimere la sua soddisfazione riguardo all’intervento eseguito;
  \item l’azienda può aumentare la sua popolarità: più recensioni positive attirano maggiormente nuovi clienti e spingono i clienti precedenti a ricontattarli;
  \item velocizza il processo di valutazione del servizio, eliminando il sistema di questionari e interviste
\end{enumerate}

\section{\href{https://www.rcsoft.it/software-assistenza-tecnica-gat/}{GAT Service}}

\begin{enumerate}
  \item GESTIONE CLIENTI: possibilità di consultare tutti gli archivi
    ad essi collegati (storico) esse siano chiamate, interventi,
    contratti, preventivi e cronologie.
  \item GESTIONE CHIAMATE: Tramite questa procedura sarà possibile
    inserire le chiamate o richieste dei clienti e fissare gli
    appuntamenti (come ad es: interventi a domicilio, riparazione in
    laboratorio).
  \item AGENDA APPUNTAMENTI: visualizza in forma grafica, e quindi
    immediata, gli appuntamenti da effettuare, da pianificare o già
    effettuati.
  \item GESTIONE INTERVENTI: A fine giornata, vengono inseriti i
    dati dell'intervento scaricando in automatico dal magazzino i
    ricambi utilizzati e generando le scadenze di pagamento
    dell'utente.
  \item GESTIONE PREVENTIVI: Per compilare i preventivi di spesa prima
    di effettuare l'intervento.  Il preventivo potrà essere effettuato
    su una chiamata specifica, per la quale verrà variato lo stato in
    preventivo effettuato.  Infine con due pulsanti si potrà
    specificare l'accettazione o il rifiuto del preventivo
    \item GESTIONE DOCUMENTI: Per l'emissione di documenti di vendita
      come fattura, fattura accompagnatoria, bolla accompagnatoria e
      vendita al banco anche tramite lettore di codici a barre.
\end{enumerate}


\section{\href{https://dynamics.microsoft.com/it-it/}{Microsoft Dynamics 365}}

Funzionalità:

\begin{enumerate}
  \item Facilità la creazione di ordini attraverso interfacce guidate

  \item Permette la gestione degli artiginai nell'agenda in modo semplice

  \item Funzione di ottimizzazione automatica nell'assegnazione dei tecnici

  \item Permette la visibilità di tutti gli ordini e il loro stato di avanzamento

  \item Permette compilazione di report guidati

  \item Pagina personalizzata per le varie categorie di lavoratori
\end{enumerate}

\section{\href{https://www.gianlucaghettini.net/app-android-per-gestione-clienti/}{App Android per gestione clienti}}

App Android per gestione clienti by Gianluca...
\begin{enumerate}
  \item Memorizzare i dati del cliente
  \item Associare categorie ai clienti, come ad esempio idraulico, elettricista
  \item Impostare scadenze per i clienti
  \item Ordinare i clienti per scadenza o nome
  \item Importazione da CSV e backup/ripristino dei dati
  \item  Ogni cliente ha la sua sezione note ed eventi: si possono aggiungere le fatture
\end{enumerate}

\section{\href{https://buffetti.it/software-gestionale/aziende/linea-azienda/manager-up/}{Buffetti - ManagerUp}}


\begin{enumerate}
\item Piattaforma gestionale client/server
\item Contabilità
\item Gestione solleciti
\item Gestione Bilancio
\item Cespiti
\item Gestione beni immobili
\item Gestione automezzi e autisti
\item Gestione Vendite
\item Gestione Acquisti
\item Gestione del Magazzino
\end{enumerate}

\section{\href{https://www.danea.it/software/easyfatt/}{Danea - Easyfatt}}

\begin{enumerate}
\item Fatturazione ricorrente
\item Scadenziario pagamenti e invio solleciti
\item Gestione documentale
\item Rapporti d'intervento
\item Analisi e report di facile consultazione
\item Rintracciabilità dei documenti
\item Trasformare un preventivo in fattura
\item Allegare documenti, foto e rapporti d'intervento in fattura
\end{enumerate}

\section{\href{https://www.planday.com/it/}{PlanDay}}



	1.Comunicazione veloce: Manda messaggi o SMS a dipendenti con la app
	2.Orologio marcatempo per rilevazione presenze: Dipendenti possono timbrare entrate e uscite dalla prima pagina della app
	3.Tabella turni dettagliata: Dipendenti possono vedere tutti i dettagli dei loro turni, direttamente sull’app
	4.Scambio turni: Effettua scambi di turno con l’approvazione del superiore
	
\section{\href{https://www.lookandgo.it/}{Look$\&$Go}}
	
1.Gestione di un numero illimitato di turni di lavoro tra più utenti e reparti con filtri dedicati alla profilazione del turnista
	2.Comunicazione tra gli Utenti (Admin di Sistema e turnisti)
	3.Statistiche strutturate per turnista, per giorno lavorativo e per turno
	4.Pianificazione turni di lavoro in modalità manuale e/o automatica
	5.Gestione delle qualifiche professionali dei turnisti e delle apparecchiature che turnano
	6.Gestione degli straordinari e visibilità a sistema dei giorni festivi

\section{https://www.fluida.io/}{Fluida}

	1.Rilevazione presenze: Rileva le presenze in sede o da remoto tramite NFC, Bluetooth e GPS.
	2.Collaboratori esterni: Stipula accordi, assegna incarichi e ricevi le fatture da freelance e collaboratori.
	3.Visualizzazione calendario: visualizzare il calendario personale o del team in formato giornaliero e settimanale, filtrare per gruppi di lavoro o per sede


        \section{ResourceGuru}
        employee scheduling software          (Employee side)
        \section{BookSteam}
        all-in-one online booking software    (Customer side)





\section{\href{openStamanager}{https://www.openstamanager.com/}}
Permette la gestione dell'assistenza tecnica, con funzionalità di gestione delle fatture.
E' disponibile nella dashboard un calendario degli interventi, ma esso non è condivisibile con i tecnici.
E' possibile effettuare statistica sui clienti/fornitori/tecnici e tutti i dati sono salvati su cloud.
La gestione degli interventi da parte dei tecnici può essere completata attraverso device come tablet.


\end{document}
